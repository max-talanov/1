%% start of file `jdoe_classic.tex'.
%% Copyright 2006 Xavier Danaux.
%
% This work may be distributed and/or modified under the
% conditions of the LaTeX Project Public License version 1.3c,
% available at http://www.latex-project.org/lppl/.

\documentclass{moderncv}
\usepackage{url}
%\usepackage{bibtopic}
%\usepackage[round]{natbib}
%\usepackage{multibib}
%\newcites{sel,all}{Selected works, Papers}

\usepackage[style=authoryear]{biblatex}
\bibliography{my_papers}
\DeclareBibliographyCategory{crucial}

% moderncv styles
%\moderncvstyle{casual}       % optional argument are 'nocolor' (black & white cv) and 'roman' (for roman fonts, instead of sans serif fonts)
\moderncvstyle{classic}       % idem

% character encoding
\usepackage[utf8]{inputenc}   % replace by the encoding you are using
%
% personal data (the given example is exhaustive; just give what you want)
\firstname{} \familyname{}
%\title{Highly-creative cognitive researcher}
%\address{Via Ciro Menotti, 15 – 47042, Cesenatico (FC), Italy}  % for classic style
%%\address{Via Ciro Menotti, 15 – 47042, Cesenatico (FC), Italy} % for casual style
\address{Max Talanov}
\phone{+7 962 571 8296} 
\email{max.talanov@gmail.com}
%\extrainfo{\url{https://www.scopus.com/authid/detail.uri?authorId=41762833600}}
%\extrainfo{\url{https://www.researchgate.net/profile/Max_Talanov}}
\extrainfo{\url{https://scholar.google.com/citations?hl=en&user=SoUgPioAAAAJ}}
%\extrainfo{\url{www.linkedin.com/in/max-talanov-a004aa16}}
%\photo[100pt]{Talanov_Max_2012} % also optional, and the optional argument is the height the picture must be resized to

%\quote{Any intelligent fool can make things bigger, more complex, and more violent. It takes a touch of genius -- and a lot of courage -- to move in the opposite direction.}% also optional
%\quote{Video meliora proboque, deteriora sequor (Ovidio, Metamorphosis, VII, 20)}
%\renewcommand{\listsymbol}{{\fontencoding{U}\fontfamily{ding}\selectfont\tiny\symbol{'102}}} % define another symbol to be used in front of the list items

% the ConTeXt symbol
%\def\ConTeXt{%
 % C%
  %\kern-.0333emo%
  %\kern-.0333emn%
  %\kern-.0667em\TeX%
  %\kern-.0333emt}

% slanted small caps (only with roman family; the sans serif font doesn't exists :-()
%\usepackage{slantsc}
%\DeclareFontFamily{T1}{myfont}{}
%\DeclareFontShape{T1}{myfont}{m}{scsl}{ <-> cork-lmssqbo8}{}
%\usefont{T1}{myfont}{m}{scsl}Testing the font

% command and color used in this document, independently from moderncv
%\definecolor{see}{rgb}{0.5,0.5,0.5}% for web links
%\newcommand{\up}[1]{\ensuremath{^\textrm{\scriptsize#1}}}% for text subscripts

%----------------------------------------------------------------------------------
%            content
%----------------------------------------------------------------------------------
\begin{document}
%\bibliographystyle{alpha}
%\maketitle
%\makequote
\makecvtitle

%\section{Personal Information}
%\cvitem{Name}{\small Max Talanov}
%\cvitem{Passport}{\small Russia}
%\cvitem{Date of birth}{\small 12 April 1974}
%\cvitem{Contacts}{\small mtalanov@it.kfu.ru (preferred), max.talanov@gmail.com}{}
%\cvitem{Research Gate}{https://www.researchgate.net/profile/Max\char`_Talanov}
 
\section{Personal Profile}

\cventry
    {Bio}{Highly-creative cognitive researcher with broad research experience}{}{}{}{Self motivated highly productive ideas generator with broad experience in: affective computing, neurobiological simulations, bio-inspired cognitive architectures, neuromorphic computing, artificial intelligence, natural language processing, probabilistic reasoning. I'm heading projects at different laboratories for 3 years managing breaking-through projects (see below).\\
      Currently I have the position of the deputy director for science of the Information Technology and Information Systems institute (ITIS) of the Kazan Federal University, where I manage research, grants and publication policies.\\
      As well as I'm the head of neurotechnology projects at neuroscience laboratory managing multidisciplinary projects in: affective computing, computational neuroscience, neuromorphic computing (electronics, memristive devices), biologically inspired cognitive architectures.}

%\cvitem{Strengths}{\small Highly creative, ideas generator, excellent at managing a research team in multidisciplinary projects, have multidisciplinary mindset in: computer science, cognitive science, neuroscience, electronics, IT, philosophy.}
%{Weaknesses}{Unable to work for a long time on a trivial tasks, has difficulties to work with demotivated team}

\section{Projects}

\cvitem{NeuCogAr (2014-now)}{\small Neurobiologically inspired cognitive architecture for simulation of neurobiologically plausible emotions (based on works of Hugo L\"{o}vheim) in a computational and robotic systems based on neural simulations. \emph{Breakthrough}: \emph{Affective computing} - first time the bio-plausible implementation of psycho-emotional states mapped to computational processes will be demonstrated. \emph{Cognitive architectures and robotics}: first time the bio-plausible emotional drives will be implemented to form behavioral strategies of an artificial system. We have already demonstrated: ``fear-like'' and ``disgust-like'' states. \emph{In collaboration with:} Jordi Vallverd\'u, Universitat Aut\'onoma de Barcelona. \href{https://github.com/research-team/neucogar}{\emph{Project site}}.}

\cvitem{Memristive brain (2016-now)}{\small The robotic part of the Robot Dream project dedicated to bio-inspired memristive implementation of a mammalian brain circuits and brain areas capable of real-time emotional processing in a robotic system, based on polyaniline memristive neurons implementation capable of inhibition and neuromodulation. \emph{Breakthrough}: \emph{Computer science, AI and robotics}: new type of hardware with new options of self-learning and adaptation real-time is implemented using bio-inspired architectures with new level of understanding of the functions of neurobiological mechanisms. \emph{Electronics}: the development of memristive direction could lead to a revolution in IT industry, triggering development of highly effective self-learning devices. \emph{Neuroscience}: the brain-computer interface has new boost with the use of memristors as the interface between living cells and not-living electronic memristors in a hybrid system. \emph{In collaboration with:} Victor Erokhin Universit\'{a} degli studi di Parma. \href{https://github.com/research-team/memristive-brain}{\emph{Project site}}.}

\cvitem{Robot Dream (2015-now)}{\small The integration of a HPC mammalian brain simulation with a real-time robotic system. Two phase architecture based on working metaphor of a mammalian dream. The ``dream phase'' consists of emotional experience collection, processing and behavioral strategy update is implemented as neural simulations on HPC cluster. The ``wake'' robotic system is based on the memristive implementation of a mammalian brain circuits, implemented in the memristive brain project. \emph{Breakthrough}: \emph{Robotics} - first time the bio-plausible emotional driven cognitive architecture integrated with robotics embodiment will be demonstrated including sensory input and motor output neural systems. \href{https://github.com/research-team/robot-dream}{\emph{Project site}}.}

\section{Granted patent(s)}

Thinking-Understanding the registration of intellectual property of software product or technology.

%Selected papers
\addtocategory{crucial}{talanov_2015,vallverdu_2016a,jordi2016importance}
\printbibliography[title={Selected papers}, category = crucial]
 
% all papers (my_papers.bib)
\nocite{*}
%\printbibliography[title={Papers}]

\end{document}

