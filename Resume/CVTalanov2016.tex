%% start of file `jdoe_classic.tex'.
%% Copyright 2006 Xavier Danaux.
%
% This work may be distributed and/or modified under the
% conditions of the LaTeX Project Public License version 1.3c,
% available at http://www.latex-project.org/lppl/.

\documentclass{moderncv}

% moderncv styles
%\moderncvstyle{casual}       % optional argument are 'nocolor' (black & white cv) and 'roman' (for roman fonts, instead of sans serif fonts)
\moderncvstyle{classic}       % idem

% character encoding
%\usepackage[utf8]{inputenc}   % replace by the encoding you are using
%
% personal data (the given example is exhaustive; just give what you want)
%\firstname{Manuel} \familyname{Mazzara}
%\title{research enthusiast}
%\address{Via Ciro Menotti, 15 – 47042, Cesenatico (FC), Italy}  % for classic style
%%\address{Via Ciro Menotti, 15 – 47042, Cesenatico (FC), Italy} % for casual style
%%\phone{+39 339 1401470} 
%\email{mazzara@libero.it}
%\extrainfo{\weblink{http://see.deri.org/zhixian}}
%\photo[100pt]{k} % also optional, and the optional argument is the height the picture must be resized to
%\quote{Any intelligent fool can make things bigger, more complex, and more violent. It takes a touch of genius -- and a lot of courage -- to move in the opposite direction.}% also optional
%\quote{Video meliora proboque, deteriora sequor (Ovidio, Metamorphosis, VII, 20)}
%\renewcommand{\listsymbol}{{\fontencoding{U}\fontfamily{ding}\selectfont\tiny\symbol{'102}}} % define another symbol to be used in front of the list items

% the ConTeXt symbol
%\def\ConTeXt{%
 % C%
  %\kern-.0333emo%
  %\kern-.0333emn%
  %\kern-.0667em\TeX%
  %\kern-.0333emt}

% slanted small caps (only with roman family; the sans serif font doesn't exists :-()
%\usepackage{slantsc}
%\DeclareFontFamily{T1}{myfont}{}
%\DeclareFontShape{T1}{myfont}{m}{scsl}{ <-> cork-lmssqbo8}{}
%\usefont{T1}{myfont}{m}{scsl}Testing the font

% command and color used in this document, independently from moderncv
%\definecolor{see}{rgb}{0.5,0.5,0.5}% for web links
%\newcommand{\up}[1]{\ensuremath{^\textrm{\scriptsize#1}}}% for text subscripts

%----------------------------------------------------------------------------------
%            content
%----------------------------------------------------------------------------------
\begin{document}
%\maketitle
%\makequote

\section{Personal Information}
\cvitem{Name}{\small Max Talanov}
\cvitem{Passport}{\small Russia}
\cvitem{Date of birth}{\small 12 April 1974}
\cvitem{Contacts}{\small mtalanov@it.kfu.ru (preferred), max.talanov@gmail.com}{}
\cvitem{Research Gate}{https://www.researchgate.net/profile/Max\char`_Talanov}
 
\section{Personal Profile}

\cvcomputer
{Strengths}{Highly-creative AI researcher/software architect and team leader with 16 years of professional experience; has experience in  management of several research and development projects in: affective computing, brain simulations, bio-electronics and machine understanding projects}
{Weaknesses}{Unable to work for a long time for a trivial tasks; difficulties to work with demotivated team}

\section{Computing Professional Experience}

\cventry{current}{Acting head of the Intellectual Robotics department in Kazan Federal University}{}{Kazan, Russian Federation}{}{Management of several research labs, scientific projects planning, funding, publications strategy, publicity and international activity management}

\cventry{current}{Head of Machine Cognition lab}{}{Kazan, Russian Federation}{}{Creation and managing scientific group, 
Leading research in several directions: affective computation, decision making, neurobiologically inspired systems, computational neurobiology, grants applications, publication activities, managing masters students, international communications and collaborations, joint lab activities}

\cventry{2015}{Lecturer in Innopolis University}{}{Kazan, Russian Federation}{}{Lecturing Affective computation course from mainly three perspectives: Philosophical (Model of six by Marvin Minsky), Psychological (Wheel of emotions by Plutchik), Neurobiological: (Cube of emotions by L\"{o}vheim)}

\cventry{2014}{Software and solution Architect in Fujitsu GDC Russia}{}{Last year I was busy with Artificial Intelligence project with specialization on Artificial Emotions, see my publications. Software Design architect in different project based on: Scala, OpenCog.RelEx, 
Neo4j, OpenCog.PLN, Stanford Parser, open NARS, MinorThird, Java, EJB, Hibernate, Spring, IBM MQ, Oracle BPEL}{}{}

\section{Education Background}

\cventry{2000}{PHD degree in Math modelling of Electromagnetic fields in plasma}{Kazan State Technological University,
Russia}{Faculty of enterprise processes management}{Thesis: \emph{Electromagnetic picture of high frequency plasma}}{}

\section{Teaching Experience}

\cvitem{2014-Now, Kazan Federal University}{\small Software Architecture}
\cvitem{2014-2015, Innopolis University}{\small Affective computing}


\section{Projects}

\cvitem{NEUCOGAR (2014-now)}{Neurobiologically inspired cognitive architecture}
\cvitem{CellCircuit (2015-now)}{Nanosensors for neural activity placed in neurons}
\cvitem{Quantitative model of emotions (2015-now)}{Quantitative model of emotions based on neuromodulation}
\cvitem{BioDynamo (2015-now)}{Simulation framework of growing brain of mammals}
\cvitem{Thinking-Understanding (2012-now)}{Intelligent system for automatic incident processing}
\cvitem{Menta (2011-2012)}{Automatic application construction via Natural Language processing}
\cvitem{IDP (2010-2011)}{Semantic Document Processing} 

\section{Language Competence}

\cvcomputer{Russian}{mother language}{English}{Full professional proficiency}


\section{Professional Training}

\cvitem{2014}{\small Teaching Excellence 2014, Carnegie Mellon University}
\cvitem{2009}{\small Software Architect}

\section{Computing Knowledge}

\cvcomputer{Operating Systems}{Microsoft Windows, Linux, Mac OS}{Programming}{Python, Java, Scala, Prolog, Refal}
\cvcomputer{Databases}{SQL, JDBC, DB2, Access, PostgreSQL, Neo4j, HypergraphDB}{Web Publishing}{HTML, XML, JavaScript, jQuery} 

\cvcomputer{Software Design and Specification}{UML, Problem Frames}{Tools}{Visio, ArgoUML, Visual Paradigm} 
\cvcomputer{Profiles}{VisualVM, .NET Profiler}{Unmanaged}{Code Analyst profiler} 


\section{Journal papers}
\cvitem{}{\small Toshchev A., Talanov M. - \emph{Computational emotional thinking and virtual neurotransmitters}, International Journal of Synthetic Emotions (IJSE), 2014, 05/2014}


\cvitem{}{\small Toshchev A., Talanov M. - \emph{Coping and High Level Emotions Aspects of Computational Emotional Thinking}, International Journal of Synthetic Emotions (IJSE) 2015, 06/2015}


\section{Conferences and Workshops}
\cvitem{}{\small Toshchev A., Talanov M. - \emph{Thinking Lifecycle as an Implementation of Machine Understanding in Software Maintenance Automation Domain}, Agent and Multi- Agent Systems: Technologies and Applications: 9th KES International Confer- ence, KES-AMSTA 2015 Sorrento, Italy, June 2015, Proceedings (Smart Innovation, Systems and Technologies), Sorento, Italy, 2015}

\cvitem{}{\small Toshchev A. - \emph{Thinking model and machine understanding in automated user request processing}, CEUR Workshop Proceedings (RCDL-2014), Dubna, Russia, 2014}

\cvitem{}{\small Toshchev A. Talanov M. - \emph{ARCHITECTURE AND REALIZATION OF IN- TELLECTUAL AGENT FOR AUTOMATIC INCIDENT PROCESSING US- ING THE ARTIFICIAL INTELLIGENCE AND SEMANTIC NETWORKS}, Electronic libraries , Kazan, Russia, 2014}

\cvitem{}{\small Toshchev A. Talanov M. Krehov A. Khasianov A. - \emph{Thinking-Understanding approach in IT maintenance domain automation}, Global Journal on Technology (WCIT-2012) , Barcelona, Spain, 2012}



\section{Conference and Workshop Activities}

\cvitem{2013}{\small AINL}
\cvitem{2015}{\small AMSTA}
\cvitem{2015}{\small AINA}
\cvitem{2015}{\small BICA}


\end{document}